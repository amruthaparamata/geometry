\let\negmedspace\undefined
\let\negthickspace\undefined
\documentclass[journal,12pt,onecolumn]{IEEEtran}
\usepackage{cite}
\usepackage{amsmath,amssymb,amsfonts,amsthm}
\usepackage{algorithmic}
\usepackage{graphicx}
\usepackage{textcomp}
\usepackage{xcolor}
\usepackage{txfonts}
\usepackage{listings}
\usepackage{enumitem}
\usepackage{mathtools}
\usepackage{gensymb}
\usepackage[breaklinks=true]{hyperref}
\usepackage{tkz-euclide} % loads  TikZ and tkz-base
\usepackage{listings}

\newcommand{\solution}{\noindent \textbf{Solution: }}
\newcommand{\cosec}{\,\text{cosec}\,}
\providecommand{\dec}[2]{\ensuremath{\overset{#1}{\underset{#2}{\gtrless}}}}
\newcommand{\myvec}[1]{\ensuremath{\begin{pmatrix}#1\end{pmatrix}}}
\newcommand{\mydet}[1]{\ensuremath{\begin{vmatrix}#1\end{vmatrix}}}
\newcommand{\myaugvec}[2]{\ensuremath{\begin{amatrix}{#1}#2\end{amatrix}}}
\providecommand{\rank}{\text{rank}}
\providecommand{\pr}[1]{\ensuremath{\Pr\left(#1\right)}}
\providecommand{\qfunc}[1]{\ensuremath{Q\left(#1\right)}}
	\newcommand*{\permcomb}[4][0mu]{{{}^{#3}\mkern#1#2_{#4}}}
\newcommand*{\perm}[1][-3mu]{\permcomb[#1]{P}}
\newcommand*{\comb}[1][-1mu]{\permcomb[#1]{C}}
\providecommand{\qfunc}[1]{\ensuremath{Q\left(#1\right)}}
\providecommand{\gauss}[2]{\mathcal{N}\ensuremath{\left(#1,#2\right)}}
\providecommand{\diff}[2]{\ensuremath{\frac{d{#1}}{d{#2}}}}
\providecommand{\myceil}[1]{\left \lceil #1 \right \rceil }
\newcommand\figref{Fig.~\ref}
\newcommand\tabref{Table~\ref}
\newcommand{\sinc}{\,\text{sinc}\,}
\newcommand{\rect}{\,\text{rect}\,}


%\bibliographystyle{ieeetr}
\begin{document}
\providecommand{\pr}[1]{\ensuremath{\Pr\left(#1\right)}}
\providecommand{\prt}[2]{\ensuremath{p_{#1}^{\left(#2\right)} }}        % own macro for this question
\providecommand{\qfunc}[1]{\ensuremath{Q\left(#1\right)}}
\providecommand{\sbrak}[1]{\ensuremath{{}\left[#1\right]}}
\providecommand{\lsbrak}[1]{\ensuremath{{}\left[#1\right.}}
\providecommand{\rsbrak}[1]{\ensuremath{{}\left.#1\right]}}
\providecommand{\brak}[1]{\ensuremath{\left(#1\right)}}
\providecommand{\lbrak}[1]{\ensuremath{\left(#1\right.}}
\providecommand{\rbrak}[1]{\ensuremath{\left.#1\right)}}
\providecommand{\cbrak}[1]{\ensuremath{\left\{#1\right\}}}
\providecommand{\lcbrak}[1]{\ensuremath{\left\{#1\right.}}
\providecommand{\rcbrak}[1]{\ensuremath{\left.#1\right\}}}
\newcommand{\sgn}{\mathop{\mathrm{sgn}}}
\providecommand{\abs}[1]{\left\vert#1\right\vert}
\providecommand{\res}[1]{\Res\displaylimits_{#1}} 
\providecommand{\norm}[1]{\left\lVert#1\right\rVert}
%\providecommand{\norm}[1]{\lVert#1\rVert}
\providecommand{\mtx}[1]{\mathbf{#1}}
\providecommand{\mean}[1]{E\left[ #1 \right]}
\providecommand{\cond}[2]{#1\middle|#2}
\providecommand{\fourier}{\overset{\mathcal{F}}{ \rightleftharpoons}}
\newenvironment{amatrix}[1]{%
  \left(\begin{array}{@{}*{#1}{c}|c@{}}
}{%
  \end{array}\right)
}
%\providecommand{\hilbert}{\overset{\mathcal{H}}{ \rightleftharpoons}}
%\providecommand{\system}{\overset{\mathcal{H}}{ \longleftrightarrow}}
	%\newcommand{\solution}[2]{\textbf{Solution:}{#1}}

%%
%	%\newcommand{\solution}[2]{\textbf{Solution:}{#1}}
%\newcommand{\solution}{\noindent \textbf{Solution: }}
%\newcommand{\cosec}{\,\text{cosec}\,}
%\numberwithin{equation}{section}
%\numberwithin{equation}{subsection}
%\numberwithin{problem}{section}
%\numberwithin{definition}{section}
%\makeatletter
%\@addtoreset{figure}{problem}
%\makeatother

%\let\StandardTheFigure\thefigure
\let\vec\mathbf

Draw the circle with centre at $\vec{O}$ and radius\\
R = OA\\
This is known as the circumradius\\
Solution:\\
Given,\\
\begin{align}
\vec{A} &= \myvec{1\\-1}
\end{align}
\begin{align}
\vec{B} &= \myvec{-4\\6}
\end{align}
\begin{align}
\vec{C} &= \myvec{-3\\-5}
\end{align}
Let AD, BE, CF are altitudes of triangle from vertices $\vec{A}$, $\vec{B}$, $\vec{C}$ respectively.\\
Equation of AD,
\begin{align}
       AD &= \myvec{1+t\\-1+\frac{t}{11}}
\end{align}
Equation of BE,
\begin{align}
       BE &= \myvec{-4+s\\6-s}
\end{align}
The point of intersection of AD and BE is $\vec{O}$;
Therefore,\\
\begin{align}
        \vec{O} &= \myvec{\frac{17}{6}\\\frac{5}{6}}
\end{align}
Radius of circle with centre $\vec{O}$
\begin{align}
             R = OA 
               = \frac{11\sqrt{22}}{6}
\end{align}
Therefore;\\
The equation of circle is
\begin{align}
\left(x - \frac{17}{6}\right)^2 + \left(y - \frac{5}{6}\right)^2 = \left(\frac{11\sqrt{22}}{6}\right)^2
\end{align}
\end{document}
